%%%%%%%%%%%%%%%%%%%%%%%%%%%%%%%%%%%%%%%%%%%%%%%%%%%%%%%%%%%%%%%%%%%%%%
%
% Documentation for the tikzlings package
% A collection of packages to draw animals in tikz
% Maintained by samcarter
%
% Project repository and bug tracker:
% https://github.com/samcarter/tikzlings
%
% Released under the LaTeX Project Public License v1.3c or later
% See http://www.latex-project.org/lppl.txt
%
% Version 0.1
% Sep 5, 2018 
%
%%%%%%%%%%%%%%%%%%%%%%%%%%%%%%%%%%%%%%%%%%%%%%%%%%%%%%%%%%%%%%%%%%%%%%
\documentclass[parskip=half]{scrartcl}

% packages %%%%%%%%%%%%%%%%%%%%%%%%%%%%%%%%%%%%%%%%%%%%%%%%%%%%%%%%%%%
\usepackage[T1]{fontenc}	
\usepackage[utf8]{inputenc}		
\usepackage[english]{babel}
\usepackage[svgnames]{xcolor}
\usepackage[bitstream-charter]{mathdesign}
\usepackage{tikzlings}
\usepackage[most]{tcolorbox}
\usepackage[paper=a4paper,margin=3cm]{geometry}
\usepackage{url}
\usepackage{xspace}
\usepackage{scrlayer-scrpage} 
\usepackage[hang,flushmargin,bottom]{footmisc}
\usepackage[%
	colorlinks=true,
	breaklinks=true,
	allcolors=SteelBlue!50!black
]{hyperref}

% macros %%%%%%%%%%%%%%%%%%%%%%%%%%%%%%%%%%%%%%%%%%%%%%%%%%%%%%%%%%%%%
\newcommand{\CTAN}{\textsc{CTAN}\xspace}
\newcommand{\TikZ}{Ti\emph{k}Z\xspace}
\newcommand{\tikzducks}{Ti\emph{k}Zducks\xspace}
\newcommand{\tikzmarmots}{Ti\emph{k}Zmarmots\xspace}
\newcommand{\tikzlings}{Ti\emph{k}Zlings\xspace}
\newcommand{\miktex}{MiK\TeX\xspace}
\newcommand{\texlive}{\TeX{}Live\xspace}

% customisation %%%%%%%%%%%%%%%%%%%%%%%%%%%%%%%%%%%%%%%%%%%%%%%%%%%%%%
\addtokomafont{sectioning}{\color{SteelBlue}}
\addtokomafont{date}{\normalsize}
\addtokomafont{author}{\normalsize}

\deftocheading{toc}{}%
\setcounter{tocdepth}{1}

\newcommand*\gobbleentrynumber[1]{}
\newcommand*\mytocformat[1]{#1}
\newcommand*\mytocpageformat[1]{#1}
\RedeclareSectionCommand[
  toclinefill=\TOCLineLeaderFill,
  tocnumwidth=0pt,
  tocentrynumberformat=\gobbleentrynumber,
  tocentryformat=\mytocformat,
  tocpagenumberformat=\mytocpageformat,
  tocbeforeskip=0.8ex plus 1pt minus 1pt
]{section}



\makeatletter
\renewcommand{\sectionlinesformat}[4]{%
\ifstr{#1}{section}{%
    \parbox[t]{\linewidth}{%
      \raggedsection\@hangfrom{\hskip #2#3}{#4}\par%
      \kern-.75\ht\strutbox\rule{\linewidth}{.8pt}%
    }%
  }{%
    \@hangfrom{\hskip #2#3}{#4}}% 
}
\makeatother

\renewcommand*{\subsectionformat}{}
\renewcommand*{\sectionformat}{}

\setlength{\footnotemargin}{0.7em}

\lstdefinestyle{duckstyle}{%
	language={[latex]TeX},
	tabsize=2,
	breaklines,
	basicstyle=\ttfamily,
	commentstyle={\color{green!50!black}\slshape}, 
	columns=fullflexible,
	alsodigit={-},
	alsoletter={3},
	emphstyle=\color{red!60!black},
	emph=[1]{
		tikzlings,
		tikzlings-marmots, tikzlings-bears, tikzlings-coatis, tikzlings-koalas, tikzlings-marmots, tikzlings-owls, tikzlings-penguins, tikzlings-snowmans,
		body, 3D, rotatehead, sideward, blush, sleeping, whiskers, teeth, shadow, askphil, leftstep, rightstep, eye, pupil, bill, feet, belly, ask, phil,
		hat, crown, tophat,  
		scale, yshift, xshift, rotate
	},
	texcsstyle=*\color{SteelBlue!50!black}\bfseries,
	keywordstyle=\color{red!60!black}\bfseries,
	morekeywords={tikzpicture},
	moretexcs={
		usepackage, usetikzlibrary, marmot, coati, bear, koala, owl, penguin, thing, tikzling, snowman
	},
	delim ={[s][\ttfamily\color{green!50!black}]{$}{$}},
	moredelim=[is][\footnotesize\ttfamily\color{orange!70!black}]{|}{|},
	index=[1][emph]
}

\tcbset{%
	colframe=SteelBlue!50!black,
	arc=0mm,
	fonttitle=\bfseries,
	sidebyside,
	listing options={style=duckstyle},
	center lower,
	righthand width=6.5cm,
	bottom=0pt, 
	top=0pt,
	tikz lower,
	height plus=3cm,
	colback=SteelBlue!30!white
}

\lstset{style=duckstyle}

\pgfmathsetseed{1}
%\setlength{\footheight}{45pt}

\cfoot{\thepage} 
\pagestyle{scrheadings}

% meta %%%%%%%%%%%%%%%%%%%%%%%%%%%%%%%%%%%%%%%%%%%%%%%%%%%%%%%%%%%%%%%
\title{The \texorpdfstring{\tikzlings}{tikzlings} package}
\subtitle{drawing animals and beings in \TikZ}
\author{%
	\texorpdfstring{\texttt{samcarter}\\[0.8em]
		\url{https://github.com/samcarter/tikzlings}
%		\url{https://www.ctan.org/pkg/tikzlings}
	}{samcarter}}
\date{Version 0.1 -- \today}

\begin{document}
\maketitle
\thispagestyle{scrheadings}

\section*{Introduction}
\label{intro}

The \tikzlings are a collection of little animals (and beings) drawn in \TikZ. It is the next evolutionary phase of the \tikzmarmots package extending it with further animals (and beings) and also adding the ability to natively use many of the accessories known from the \tikzducks package. 

This package is work in progress, therefore I would be happy to hear your feedback and ideas how to improve the package. 
The head version of the source code can be found on \url{github.com/samcarter/tikzlings}, including a bug tracker -- please make constructive use of it! 
%A more stable package version can be found on \CTAN (\url{www.ctan.org/pkg/tikzlings}) and is included in both \miktex and \texlive as \tikzlings. 

\subsection*{Acknowledgements}

I'd like to thank the friendly and helpful community of \href{https://tex.stackexchange.com/}{TeX.Stackexchange} for their suggestions, feedback and help to create this package and find a suitable name for it. As a thank you all the \tikzlings have a name which is in some way or another connected to the users of TeX.SE.

\subsection*{License}

Copyright \raisebox{0.2em}{\tiny\fontfamily{cmr}\selectfont\textcopyright}
\texttt{samcarter}. Permission is granted to copy, distribute and\slash or modify this software under the terms of the LaTeX project public licence, version 1.3c or later \url{http://www.latex-project.org/lppl.txt}.

\clearpage
\section*{The \tikzlings}

The \tikzlings package is a collection of packages. It can either be loaded as a whole with \lstinline|\usepackage{tikzlings}| or the subpackages containing the individual animals (and beings) can be used separately, for example by loading \lstinline|\usepackage{tikzlings-marmots}|.

The basic usage is the same for all animals (and beings). Inside a \lstinline|tikzpicture|, the \tikzlings can be added via \color{SteelBlue!50!black}\lstinline|\<name_of_the_tikzling>|\color{black}. For example

\begin{tcblisting}{}
\marmot
\end{tcblisting}

will produce a marmot. All usual \TikZ keys can be passed as optional argument to change the appearance. For example scaling and rotating the \tikzlings can be done by

\begin{tcblisting}{}
\coati[rotate=30,scale=0.5]
\end{tcblisting}

In addition to the standard options provided by \TikZ each \tikzlings also comes with some addition options which are listed in the following sections. If these additional options consist of multiple words they are available both with and without spaces, for example \lstinline|askphil| and \lstinline|ask phil| will be treated as the same.

\subsection*{List of all \tikzlings:}

\begingroup
	\hypersetup{hidelinks}
	\tableofcontents
\endgroup

%%%%%%%%%%%%%%%%%%%%%%%%%%%%%%%%%%%%%%%%%%%%%%%%%%%%%%%%%%%%%%%%%%%%%%

\clearpage
\section[Bear]{B\"ar, the teddy bear}

\subsection{Package name}

\begin{tcolorbox}[lower separated=false, lefthand width=.8\linewidth]
\vspace*{0.5cm}
\lstinline|\usepackage{tikzlings-bears}|
\vspace*{0.5cm}
\end{tcolorbox}

\subsection{Basic Usage}

\begin{tcblisting}{}
\bear
\end{tcblisting}

\subsection{Options}

The basic teddy bear can be modified by changing it's colour

\begin{tcblisting}{}
\bear[body=SteelBlue]
\end{tcblisting}

The key \lstinline|3D| will make the teddy bear 3-dimensional:

\begin{tcblisting}{}
\bear[3D]
\end{tcblisting}

%%%%%%%%%%%%%%%%%%%%%%%%%%%%%%%%%%%%%%%%%%%%%%%%%%%%%%%%%%%%%%%%%%%%%%

\clearpage
\section[Coati]{007, the coati}

\subsection{Package name}

\begin{tcolorbox}[lower separated=false, lefthand width=.8\linewidth]
\vspace*{0.5cm}
\lstinline|\usepackage{tikzlings-coatis}|
\vspace*{0.5cm}
\end{tcolorbox}

\subsection{Basic Usage}

\begin{tcblisting}{}
\coati
\end{tcblisting}

\subsection{Options}

The basic coati can be modified by changing it's colour
\begin{tcblisting}{}
\coati[body=SteelBlue]
\end{tcblisting}

The head of the \lstinline|\coati| can be rotated, but please don't overdo this, otherwise his neck might break.
\begin{tcblisting}{}
\coati[rotatehead=-15]
\end{tcblisting}

For the head an alternative sidewards facing head is available. It can be combined with the \lstinline|rotatehead| option.
\begin{tcblisting}{}
\coati[sideward]
\end{tcblisting}

Finally the key \lstinline|3D| will make the coati 3-dimensional:
\begin{tcblisting}{}
\coati[3D]
\end{tcblisting}

%%%%%%%%%%%%%%%%%%%%%%%%%%%%%%%%%%%%%%%%%%%%%%%%%%%%%%%%%%%%%%%%%%%%%%

\clearpage
\section[Koala]{Will, the koala}

\subsection{Package name}

\begin{tcolorbox}[lower separated=false, lefthand width=.8\linewidth]
\vspace*{0.5cm}
\lstinline|\usepackage{tikzlings-koalas}|
\vspace*{0.5cm}
\end{tcolorbox}

\subsection{Basic Usage}

\begin{tcblisting}{}
\koala
\end{tcblisting}

\subsection{Options}

The basic koala can be modified by changing it's colour

\begin{tcblisting}{}
\koala[body=SteelBlue]
\end{tcblisting}

making it blush

\begin{tcblisting}{}
\koala[blush]
\end{tcblisting}

If it is tired, it can sleep:

\begin{tcblisting}{}
\koala[sleeping]
\end{tcblisting}

Finally the key \lstinline|3D| will make the koala 3-dimensional:

\begin{tcblisting}{}
\koala[3D]
\end{tcblisting}

%%%%%%%%%%%%%%%%%%%%%%%%%%%%%%%%%%%%%%%%%%%%%%%%%%%%%%%%%%%%%%%%%%%%%%

\clearpage
\section[Marmot]{Phil, the marmot}

\subsection{Package name}

\begin{tcolorbox}[lower separated=false, lefthand width=.8\linewidth]
\vspace*{0.5cm}
\lstinline|\usepackage{tikzlings-marmots}|
\vspace*{0.5cm}
\end{tcolorbox}

\subsection{Basic Usage}

\begin{tcblisting}{}
\marmot
\end{tcblisting}

\subsection{Options}

The basic marmot can be modified by changing it's colour

\begin{tcblisting}{}
\marmot[body=SteelBlue]
\end{tcblisting}

making it blush

\begin{tcblisting}{}
\marmot[blush]
\end{tcblisting}

or adding whiskers

\begin{tcblisting}{}
\marmot[whiskers=gray]
\end{tcblisting}

Some marmots even show their chisel teeth:

\begin{tcblisting}{}
\marmot[teeth=white]
\end{tcblisting}

The marmots also can cast a shadow:

\begin{tcblisting}{}
\marmot[shadow]
\end{tcblisting}

This ability is important if you want to ask Punxsutawney Phil\footnote{\url{https://en.wikipedia.org/wiki/Punxsutawney_Phil}} on Groundhog Day how the weather is going to be. With a probability derived from the statistics of 120 Groundhog Days\footnote{\url{https://www.livescience.com/32974-punxsutawney-phil-weather-prediction-accuracy.html}} the option \lstinline|askphil| might or might not result in a shadow.

\begin{tcblisting}{}
\marmot[askphil]
\end{tcblisting}

If a good weather prognosis is derived, the happy marmot can dance by lifting up it's left and right foot: 

\begin{tcblisting}{}
\marmot[leftstep]
\marmot[rightstep,xshift=2cm]
\end{tcblisting}

Finally the key \lstinline|3D| will make the marmot 3-dimensional:

\begin{tcblisting}{}
\marmot[3D]
\end{tcblisting}

%%%%%%%%%%%%%%%%%%%%%%%%%%%%%%%%%%%%%%%%%%%%%%%%%%%%%%%%%%%%%%%%%%%%%%

\clearpage
\section[Owl]{Jake, the owl}

\subsection{Package name}

\begin{tcolorbox}[lower separated=false, lefthand width=.8\linewidth]
\vspace*{0.5cm}
\lstinline|\usepackage{tikzlings-owls}|
\vspace*{0.5cm}
\end{tcolorbox}

\subsection{Basic Usage}

\begin{tcblisting}{}
\owl
\end{tcblisting}

\subsection{Options}

The basic owl can be modified by changing it's colour

\begin{tcblisting}{}
\owl[body=SteelBlue]
\end{tcblisting}

In addition to the colour of the body, the colour of various body parts can be adjusted. In detail these are the eyes:

\begin{tcblisting}{}
\owl[eye=red]
\end{tcblisting}

the pupils:

\begin{tcblisting}{}
\owl[pupil=red]
\end{tcblisting}

the bill:

\begin{tcblisting}{}
\owl[bill=red]
\end{tcblisting}

and the feet:

\begin{tcblisting}{}
\owl[feet=red]
\end{tcblisting}

Finally the key \lstinline|3D| will make the owl 3-dimensional:

\begin{tcblisting}{}
\owl[3D]
\end{tcblisting}

%%%%%%%%%%%%%%%%%%%%%%%%%%%%%%%%%%%%%%%%%%%%%%%%%%%%%%%%%%%%%%%%%%%%%%

\clearpage
\section[Penguin]{Tux, the penguin}

\subsection{Package name}

\begin{tcolorbox}[lower separated=false, lefthand width=.8\linewidth]
\vspace*{0.5cm}
\lstinline|\usepackage{tikzlings-penguins}|
\vspace*{0.5cm}
\end{tcolorbox}

\subsection{Basic Usage}

\begin{tcblisting}{}
\penguin
\end{tcblisting}

\subsection{Options}

The basic penguin can be modified by changing it's colour

\begin{tcblisting}{}
\penguin[body=SteelBlue]
\end{tcblisting}

In addition to the colour of the body, the colour of various body parts can be adjusted. In detail these are the eyes:

\begin{tcblisting}{}
\penguin[eye=red]
\end{tcblisting}

the pupils:

\begin{tcblisting}{}
\penguin[pupil=red]
\end{tcblisting}

the bill:

\begin{tcblisting}{}
\penguin[bill=red]
\end{tcblisting}

the belly:

\begin{tcblisting}{}
\penguin[belly=red]
\end{tcblisting}

and the feet:

\begin{tcblisting}{}
\penguin[feet=red]
\end{tcblisting}

Finally the key \lstinline|3D| will make the penguin 3-dimensional:

\begin{tcblisting}{}
\penguin[3D]
\end{tcblisting}


%%%%%%%%%%%%%%%%%%%%%%%%%%%%%%%%%%%%%%%%%%%%%%%%%%%%%%%%%%%%%%%%%%%%%%

\clearpage
\section[Snowman]{Yuki, the snowman}

\subsection{Package name}

\begin{tcolorbox}[lower separated=false, lefthand width=.8\linewidth]
\vspace*{0.5cm}
\lstinline|\usepackage{tikzlings-snowmans}|
\vspace*{0.5cm}
\end{tcolorbox}

\subsection{Basic Usage}

\begin{tcblisting}{}
\snowman
\end{tcblisting}

\subsection{Options}

The basic snowman can be modified by changing it's colour

\begin{tcblisting}{}
\snowman[body=SteelBlue]
\end{tcblisting}

In addition to the colour of the body, the colour of various body parts can be adjusted. In detail these are the eyes:

\begin{tcblisting}{}
\snowman[eye=red]
\end{tcblisting}

the nose:

\begin{tcblisting}{}
\snowman[nose=red]
\end{tcblisting}

the mouth:

\begin{tcblisting}{}
\snowman[mouth=red]
\end{tcblisting}

and the buttons:

\begin{tcblisting}{}
\snowman[buttons=red]
\end{tcblisting}

Finally the key \lstinline|3D| will make the snowman 3-dimensional:

\begin{tcblisting}{}
\snowman[3D]
\end{tcblisting}

\clearpage
\section[Ti\emph{k}Zlings]{some name, the tikzling}

\subsection{Package name}

\begin{tcolorbox}[lower separated=false, lefthand width=.8\linewidth]
\vspace*{0.5cm}
\lstinline|\usepackage{tikzlings}|
\vspace*{0.5cm}
\end{tcolorbox}

\subsection{Basic Usage}

\begin{tcblisting}{}
\tikzling
\end{tcblisting}

\subsection{Options}

Only options common for all \tikzlings are supported for the \lstinline|\tikzling|. At the moment these are: 

\begin{tcblisting}{}
\tikzling[body=SteelBlue]
\end{tcblisting}

The key \lstinline|3D| will make the Ti\emph{k}Zlings 3-dimensional:

\begin{tcblisting}{}
\tikzling[3D]
\end{tcblisting}

\clearpage
\section{Accessories}

To customise the \tikzlings the package provides a number of accessories which can be added to all the \tikzlings simply by adding the respective keyword as optional argument:

\begin{tcblisting}{}
\bear[hat]
\end{tcblisting}
 
For most of these items, the colour can be customised:

\begin{tcblisting}{}
\koala[crown=orange!50!yellow]
\end{tcblisting}

Unfortunately it is very difficult to create accessories that will fit all the different shapes of the \tikzlings. Therefore it is also possible to add them separately as optional argument of the \lstinline|\thing| macro, which allows more control of their size and placement:

\begin{tcblisting}{}
\owl
\thing[tophat,scale=1.5,yshift=-0.6cm,xshift=-0.05cm]
\end{tcblisting}

\subsection{List of available accessories}

\begin{tcblisting}{}
\snowman[tophat]
\end{tcblisting}

\begin{tcblisting}{}
\owl[hat]
\end{tcblisting}

\begin{tcblisting}{}
\koala[alien]
\end{tcblisting}

\begin{tcblisting}{}
\penguin[crown]
\end{tcblisting}

\begin{tcblisting}{}
\marmot[crystalball]
\end{tcblisting}

\end{document}
